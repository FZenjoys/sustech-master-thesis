% !TeX root = ./sustechthesis-example.tex

% 论文基本信息配置

\thusetup{
  %******************************
  % 注意:
  %   1. 配置里面不要出现**空行**
  %   2. 不需要的配置信息可以删除
  %   3. 建议先阅读文档中所有关于选项的说明
  %******************************
  %
  % 输出格式
  %   选择打印版(print)或用于提交的电子版(electronic),前者会插入空白页以便直接双面打印
  %
  output = electronic,
  %
  % 标题
  %   可使用“\\”命令手动控制换行
  %   如果需要使用副标题,取消 subtitle 和 subtitle* 的注释即可(\color{gray} 仅作展示用,使用时请删去)。
  %
  title  = {南方科技大学学位论文 \LaTeX{} 模板 (Support English) 使用示例文档 v\version},
  title* = {An Introduction to \LaTeX{} Thesis Template of Southern University of Science and Technology v\version},
  % subtitle = {\color{gray} 可选的副标题可选的副标题可选的副标题可选的副标题可选的副标题可选的副标题},
  % subtitle* = {\color{gray} optional subtitle optional subtitle optional subtitle optional subtitle optional subtitle optional subtitle},
  %
  % 学位
  %   1. 学术型
  %      - 中文
  %        需注明所属的学科门类,例如:
  %        哲学、经济学、法学、教育学、文学、历史学、理学、工学、农学、医学、
  %        军事学、管理学、艺术学
  %      - 英文
  %        博士:Doctor of Philosophy
  %        硕士:
  %          哲学、文学、历史学、法学、教育学、艺术学门类,公共管理学科
  %          填写“Master of Arts“,其它填写“Master of Science”
  %   2. 专业型
  %      直接填写专业学位的名称,例如:
  %      教育博士、工程硕士等
  %      Doctor of Education, Master of Engineering
  %
  degree-domain = {工学}, % 【中文】学科门类:可选理学、工学
  degree-domain* = {Engineering}, % 【英文】学位等级:可选Science, Engineering
  gongsuo = false, % 是否为工程硕士:工程硕士则填 true ,学术硕士或其他为 false 。
  %
  % 培养单位
  %   填写所属院系的全名
  %   超长英文系名可以手动换行
  department = {计算机科学与技术系},
  department* = {School of System Design and \\Intelligent Manufacturing},
  %
  % 学科
  %   1. 学术型学位
  %      获得一级学科授权的学科填写一级学科名称,其他填写二级学科名称
  %   2. 工程硕士
  %      工程领域名称
  %
  discipline  = {计算机科学与工程},
  discipline* = {Computer Science and Engineering},
  %
  % 姓名
  %
  author  = {李子强},
  author* = {Li Ziqiang},
  %
  % 指导教师
  %   中文姓名和职称之间以英文逗号“,”分开,下同
  %
  supervisor  = {某某某(Alice Bob)助理教授},
  supervisor* = {Assistant Professor Alice Bob},
  %
  % 日期
  %   使用 ISO 格式;默认为当前时间
  %   date 为第一页全中文大写日期,defense-date 为第二、三页的答辩日期。
  %   需要按 {年-月-日} 格式填写,如不显示“日”,可以随意填一个日期,但是不能为空。
  %
  date = {2010-12-20},
  defense-date = {2020-12-20},
  %
  % 密级
  %   公开, 秘密, 机密, 绝密
  %
  statesecrets={公开},
  %
  % 国内图书分类号,国际图书分类号
  %
  natclassifiedindex={TM301.2},
  intclassifiedindex={62-5},
}

% 载入所需的宏包

% 可以使用 nomencl 生成符号和缩略语说明
% \usepackage{nomencl}
% \makenomenclature

% 表格加脚注
\usepackage{threeparttable}

% 表格中支持跨行
\usepackage{multirow}

% 量和单位
\usepackage{siunitx}

% 定理类环境宏包
\usepackage{amsthm}
% 也可以使用 ntheorem
% \usepackage[amsmath,thmmarks,hyperref]{ntheorem}

% 参考文献使用 BibTeX + natbib 宏包
% 顺序编码制
\usepackage[sort]{natbib}
\bibliographystyle{thuthesis-numeric}

% % 参考文献使用 BibLaTeX 宏包
% \usepackage[backend=biber,style=thuthesis-numeric]{biblatex}
% % 声明 BibLaTeX 的数据库
% \addbibresource{ref/refs.bib}

% 定义所有的图片文件在 figures 子目录下
\graphicspath{{figures/}}

% 数学命令
\newcommand\dif{\mathop{}\!\mathrm{d}}  % 微分符号

% hyperref 宏包在最后调用
\usepackage{hyperref}
\usepackage{ragged2e}

% 固定宽度的表格。放在 hyperref 之前的话,tabularx 里的 footnote 显示不出来。
\usepackage{tabularx}

% 跨页表格,必须在 hyperref 之后使用否则会报错。
\usepackage{longtable}
