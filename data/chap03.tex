% !TeX root = ../sustechthesis-example.tex

\chapter{数学符号和公式}

\section{数学符号}

研究生《写作指南》要求量及其单位所使用的符号应符合国家标准《国际单位制及其应用》(GB 3100—1993)、《有关量、单位和符号的一般原则》(GB/T 3101—1993) 的规定。
模板中使用 \pkg{unicode-math} 宏包来配置数学符号,
与 \LaTeX{} 默认的英美国家的符号习惯有所差异:
\begin{enumerate}
  \item 大写希腊字母默认为斜体,如 \cs{Delta}:$\Delta$。
  \item 有限增量符号 $\increment$(U+2206)应使用 \pkg{unicode-math} 宏包提供的
    \cs{increment} 命令。
  \item 向量、矩阵和张量要求粗斜体,应该使用 \pkg{unicode-math} 的 \cs{symbf} 命令,
    如 \verb|\symbf{A}|、\verb|\symbf{\alpha}|。
  \item 数学常数和特殊函数要求用正体,应使用 \cs{symup} 命令,
    如 $\symup{\pi} = 3.14\dots$; $\symup{e} = 2.718\dots$,
  \item 微分号和积分号使用使用正体,比如 $\int f(x) \dif x$。
\end{enumerate}

关于数学符号更多的用法,参考
\href{http://mirrors.ctan.org/macros/latex/contrib/unicode-math/unicode-math.pdf}{\pkg{unicode-math}}
宏包的使用说明,
全部数学符号命的令参考
\href{http://mirrors.ctan.org/macros/latex/contrib/unicode-math/unimath-symbols.pdf}{\pkg{unimath-symbols}}。

关于量和单位推荐使用
\href{http://mirrors.ctan.org/macros/latex/contrib/siunitx/siunitx.pdf}{\pkg{siunitx}}
宏包,
可以方便地处理希腊字母以及数字与单位之间的空白,
比如:
\SI{6.4e6}{m},
\SI{9}{\micro\meter},
\si{kg.m.s^{-1}},
\SIrange{10}{20}{\degreeCelsius}。



\section{数学公式}

数学公式可以使用 \env{equation} 和 \env{equation*} 环境。
注意数学公式的引用应前后带括号,建议使用 \cs{eqref} 命令,比如式 \eqref{eq:example}。
\begin{equation}
  \frac{1}{2 \symup{\pi} \symup{i}} \int_\gamma f = \sum_{k=1}^m n(\gamma; a_k) \mathscr{R}(f; a_k)
  \label{eq:example}
\end{equation}
注意公式编号的引用应含有圆括号,可以使用 \cs{eqref} 命令。

多行公式尽可能在“=”处对齐,推荐使用 \env{align} 环境。
\begin{align}
  a & = b + c + d + e \\
    & = f + g
\end{align}



\section{数学定理}

定理环境的格式可以使用 \pkg{amsthm} 或者 \pkg{ntheorem} 宏包配置。
用户在导言区载入这两者之一后,模板会自动配置 \env{thoerem}、\env{proof} 等环境。

\begin{theorem}[Lindeberg--Lévy 中心极限定理]
  设随机变量 $X_1, X_2, \dots, X_n$ 独立同分布, 且具有期望 $\mu$ 和有限的方差 $\sigma^2 \ne 0$,
  记 $\bar{X}_n = \frac{1}{n} \sum_{i+1}^n X_i$,则
  \begin{equation}
    \lim_{n \to \infty} P \left(\frac{\sqrt{n} \left( \bar{X}_n - \mu \right)}{\sigma} \le z \right) = \Phi(z),
  \end{equation}
  其中 $\Phi(z)$ 是标准正态分布的分布函数。
\end{theorem}
\begin{proof}
  Trivial.
\end{proof}

同时模板还提供了 \env{assumption}、\env{definition}、\env{proposition}、
\env{lemma}、\env{theorem}、\env{axiom}、\env{corollary}、\env{exercise}、
\env{example}、\env{remar}、\env{problem}、\env{conjecture} 这些相关的环境。
