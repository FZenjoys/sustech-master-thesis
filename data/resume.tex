% !TeX root = ../sustechthesis-example.tex

\begin{resume}

  \section*{个人简历}
  % \section*{Resume}

  ××××年××月××日出生于××××。

  ××××年××月考入××大学××院(系)××专业,××××年××月本科毕业并获得××学学士学位。

  ××××年××月——××××年××月,在××大学××院(系)××学科学习并获得××学硕士学位。

  获奖情况:如获三好学生、优秀团干部、×奖学金等(不含科研学术获奖)。

  工作经历:……

  \section*{在学期间完成的相关学术成果(无学术成果时此项不必列出)}
  % \section*{Academic Achievements during the Study for an Academic Degree}

  特别注意,下面的引用文献部分需要使用半角括号,例如[J],(已被xxxx录用)。(本行在使用时请删除)。

  \subsection*{学术论文}
  % \subsection*{Academic Articles}

  \begin{achievements}
    \item Pei S, Huang L L, Li G, et al. Magnetic Raman continuum in single-crystalline $\mathrm{H_3LiIr_2O_6}$[J]. Physical Review B, 2020, 101(20): 201101. (SCI收录, IDS号为LJ4UN, IF=3. 575, 对应学位论文2.2节和第5章.)
    \item Pei S, Tang J, Liu C, et al. Orbital-fluctuation freezing and magnetic-nonmagnetic phase transition in $\mathrm{α-TiBr_3}$[J]. Applied Physics Letters, 2020, 117(13): 133103. (SCI收录, IDS号为NY3GK, IF=3. 597, 对应学位论文2.2节和第3章.)
  \end{achievements}

  \subsection*{申请及已获得的专利(无专利时此项不必列出)}
  % \subsection*{Patents}

  \begin{achievements}
    \item 任天令, 杨轶, 朱一平, 等. 硅基铁电微声学传感器畴极化区域控制和电极连接的方法: 中国, CN1602118A[P]. 2005-03-30.
    \item Ren T L, Yang Y, Zhu Y P, et al. Piezoelectric micro acoustic sensor based on ferroelectric materials: USA, No.11/215, 102[P]. (美国发明专利申请号.)
  \end{achievements}

  \subsection*{参与的科研项目及获奖情况(无获奖时此项不必列出)}
  \begin{achievements}
    \item 姜锡洲,×××××研究,××省自然科学基金项目。课题编号:××××,长长长长长长长长长长长长长长长长长长长长长长长长长长长长长长长长长长长长长长长长长长长。
    \item ×××,×××××研究,××省自然科学基金项目。课题编号:××××。
    \item ×××,×××××研究,××省自然科学基金项目。课题编号:××××。
  \end{achievements}

\end{resume}
