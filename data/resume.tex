% !TeX root = ../sustechthesis-example.tex

\begin{resume}

  \section*{个人简历}
  % \section*{Resume}

  ××××年××月××日出生于××××。

  ××××年××月考入××大学××院(系)××专业,××××年××月本科毕业并获得××学学士学位。

  ××××年××月——××××年××月,在××大学××院(系)××学科学习并获得××学硕士学位。

  获奖情况:如获三好学生、优秀团干部、×奖学金等(不含科研学术获奖)。

  工作经历:……

  \section*{在学期间完成的相关学术成果}
  % \section*{Academic Achievements during the Study for an Academic Degree}

  “学术论文”和“申请及已获得的专利”使用 \env{localref} 环境生成,环境内使用 \cs{nocite}\{\texttt{<citekey>}\} 引用参考文献
  (与 \cs{cite}\{\texttt{<citekey>}\} 使用方式一致)。《指南》要求引用编号连续,故自行计算起始编号作为 \env{localref} 环境参数。“参与的科研项目及获奖情况”使用 \env{achievements} 环境生成,使用方式同 \env{enumerate} 环境,同样需要指定起始编号以满足《指南》要求。\textbf{本段在使用时请删除}。

  \subsection{学术论文}
  % \subsection{Academic papers}

  % \begin{localref}{参考文献路径}[起始序号(可选)]
  %   \nocite{引用条目代号} %% 使用方式同 \cite{}
  % \end{localref}

  \begin{localref}{ref/refs}
    \nocite{zhangkun1994}
    \nocite{bixon1996dynamics}
    \nocite{kamiya2018nature}
  \end{localref}


  \subsection{申请及已获得的专利(无获奖时此项不必列出)}
  % \subsection{Patents}

  % \begin{localref}{参考文献路径}[起始序号(可选)]
  %   \nocite{引用条目代号} %% 使用方式同 \cite{}
  % \end{localref}

  \begin{localref}{ref/refs}[4]
    \nocite{chenxu2020yejing}
    \nocite{jiangxizhou1980}
  \end{localref}

  \subsection{参与的科研项目及获奖情况(无获奖时此项不必列出)}
  \begin{achievements}[start=6]
    \item 姜锡洲,×××××研究,××省自然科学基金项目。课题编号:××××,长长长长长长长长长长长长长长长长长长长长长长长长长长长长长长长长长长长长长长长长长长长。
    \item ×××,×××××研究,××省自然科学基金项目。课题编号:××××。
    \item ×××,×××××研究,××省自然科学基金项目。课题编号:××××。
  \end{achievements}

\end{resume}
