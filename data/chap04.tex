% !TeX root = ../sustechthesis-example.tex

\chapter{引用文献的标注}

模板支持 BibTeX 和 BibLaTeX 两种方式处理参考文献。
下文主要介绍 BibTeX 配合 \pkg{natbib} 宏包的主要使用方法。


\section{顺序编码制}

依学校样式规定,一般使用 \cs{cite},即序号置于方括号中,引文页码会放在括号外。统一处引用的连续序号会自动用短横线连接。

如多次引用同一文献,可能需要标注页码,例如:引用第二页\cite[2]{zhangkun1994},引用第五页\cite[5]{zhangkun1994}。

\thusetup{
  cite-style = super,
}
\begin{tabular}{l@{\quad$\Rightarrow$\quad}l}
  \verb|\cite{zhangkun1994}|               & \cite{zhangkun1994}   {\kaishu 不带页码的上标引用}            \\
  \verb|\cite[42]{zhangkun1994}|           & \cite[42]{zhangkun1994} {\kaishu 手动带页码的上标引用}          \\
  \verb|\cite{zhangkun1994,zhukezhen1973}| & \cite{zhangkun1994,zhukezhen1973}  {\kaishu 一次多篇文献的上标引用}  \\
\end{tabular}

注意,引文参考文献的每条都要在正文中标注
\cite{zhangkun1994,zhukezhen1973,dupont1974bone,zhengkaiqing1987,%
  jiangxizhou1980,jianduju1994,merkt1995rotational,mellinger1996laser,%
  bixon1996dynamics,mahui1995,carlson1981two,taylor1983scanning,%
  taylor1981study,shimizu1983laser,atkinson1982experimental,%
  kusch1975perturbations,guangxi1993,huosini1989guwu,wangfuzhi1865songlun,%
  zhaoyaodong1998xinshidai,biaozhunhua2002tushu,chubanzhuanye2004,%
  who1970factors,peebles2001probability,baishunong1998zhiwu,%
  weinstein1974pathogenic,hanjiren1985lun,dizhi1936dizhi,%
  tushuguan1957tushuguanxue,aaas1883science,fugang2000fengsha,%
  xiaoyu2001chubanye,oclc2000about,scitor2000project%
}。

引用测试:2个连续引用\cite{zhangkun1994,zhukezhen1973},2个间隔\cite{zhangkun1994,dupont1974bone},3个连续\cite{zhangkun1994,zhukezhen1973,dupont1974bone}。

\subsection{支持三级目录显示}

支持三级目录显示